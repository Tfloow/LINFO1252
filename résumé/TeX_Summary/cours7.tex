\section{Cours 7}\label{cours-7}

\subsection{Sémaphores et Compléments sur les
Threads}\label{suxe9maphores-et-compluxe9ments-sur-les-threads}

\subsubsection{Sémaphore}\label{suxe9maphore}

Une sémaphore est donc une structure qui contient:

\begin{itemize}
\tightlist
\item
  Entier non-signé: positif ou nul
\item
  File vers les pointeurs de threads en attente (\emph{Blocked})
\end{itemize}

\begin{Shaded}
\begin{Highlighting}[]
\PreprocessorTok{\#include }\ImportTok{\textless{}semaphore.h\textgreater{}}\PreprocessorTok{ }
\DataTypeTok{int}\NormalTok{ sem\_init}\OperatorTok{(}\NormalTok{sem\_t }\OperatorTok{*}\NormalTok{sem}\OperatorTok{,} \DataTypeTok{int}\NormalTok{ pshared}\OperatorTok{,} \DataTypeTok{unsigned} \DataTypeTok{int}\NormalTok{ value}\OperatorTok{);} \CommentTok{// initialise avec value le nombre de place}
\CommentTok{//pshared = 0}
\DataTypeTok{int}\NormalTok{ sem\_destroy}\OperatorTok{(}\NormalTok{sem\_t }\OperatorTok{*}\NormalTok{sem}\OperatorTok{);} \CommentTok{// destruction du semaphore }
\DataTypeTok{int}\NormalTok{ sem\_wait}\OperatorTok{(}\NormalTok{sem\_t }\OperatorTok{*}\NormalTok{sem}\OperatorTok{);} \CommentTok{// Attente place}
\DataTypeTok{int}\NormalTok{ sem\_post}\OperatorTok{(}\NormalTok{sem\_t }\OperatorTok{*}\NormalTok{sem}\OperatorTok{);} \CommentTok{// Liberation place}
\end{Highlighting}
\end{Shaded}

Utile pour les problèmes de rendez-vous, producteurs-consommateurs,
lecteurs-écrivains, \ldots{}

Donc une sémaphore avec \texttt{value\ =\ 1} est une sorte de mutex.
Attention, un autre thread peut libérer notre sémaphore ce qui est
différent d'un mutex où c'est généralement le même thread appelle
\texttt{lock()} et \texttt{unlock()}.

\paragraph{Problème du Rendez-Vous}\label{probluxe8me-du-rendez-vous}

2 phases:

\begin{enumerate}
\def\labelenumi{\arabic{enumi}.}
\tightlist
\item
  N threads s'exécutent indépendamment
\item
  Commence quand tous les threads ont finit la phase 1. (on peut
  utiliser la structure \texttt{pthread\_barrier\_init})
\end{enumerate}

\begin{Shaded}
\begin{Highlighting}[]
\NormalTok{sem\_t rendezvous}\OperatorTok{;} \CommentTok{// Bloque les threads en attente de RDV }
\NormalTok{pthread\_mutex\_t mutex}\OperatorTok{;} \CommentTok{// Protège accès à count}
\DataTypeTok{int}\NormalTok{ count}\OperatorTok{=}\DecValTok{0}\OperatorTok{;} \CommentTok{// Nombre de threads ayant atteint le point de RDV}

\NormalTok{sem\_init}\OperatorTok{(\&}\NormalTok{rendezvous}\OperatorTok{,}\DecValTok{0}\OperatorTok{,}\DecValTok{0}\OperatorTok{)}

\NormalTok{premiere\_phase}\OperatorTok{();} 

\CommentTok{// section critique }
\NormalTok{pthread\_mutex\_lock}\OperatorTok{(\&}\NormalTok{mutex}\OperatorTok{);}\NormalTok{ count}\OperatorTok{++;} 
\ControlFlowTok{if}\OperatorTok{(}\NormalTok{count}\OperatorTok{==}\NormalTok{N}\OperatorTok{)} \OperatorTok{\{}
    \CommentTok{// tous les threads sont arrivés }
\NormalTok{    sem\_post}\OperatorTok{(\&}\NormalTok{rendezvous}\OperatorTok{);}
\OperatorTok{\}}
\NormalTok{pthread\_mutex\_unlock}\OperatorTok{(\&}\NormalTok{mutex}\OperatorTok{);} 
\CommentTok{// attente à la barrière }
\NormalTok{sem\_wait}\OperatorTok{(\&}\NormalTok{rendezvous}\OperatorTok{);}
\CommentTok{// libération d\textquotesingle{}un autre thread en attente }
\NormalTok{sem\_post}\OperatorTok{(\&}\NormalTok{rendezvous}\OperatorTok{);}

\NormalTok{seconde\_phase}\OperatorTok{();}
\end{Highlighting}
\end{Shaded}

\paragraph{Producteurs-Consommateurs}\label{producteurs-consommateurs}

On a un ensemble de threads qui produisent quelque chosent et d'autres
qui vont le consommer. Les producteurs vont donc placer leurs données
dans une zone partagée et les consommateurs vont pouvoir se servir.

\begin{Shaded}
\begin{Highlighting}[]
\CommentTok{// Initialisation }
\PreprocessorTok{\#define N }\DecValTok{10}\PreprocessorTok{ }
\CommentTok{// places dans le buffer }
\NormalTok{pthread\_mutex\_t mutex}\OperatorTok{;} 
\NormalTok{sem\_t empty}\OperatorTok{;} 
\NormalTok{sem\_t full}\OperatorTok{;}

\NormalTok{pthread\_mutex\_init}\OperatorTok{(\&}\NormalTok{mutex}\OperatorTok{,}\NormalTok{ NULL}\OperatorTok{);} 
\NormalTok{sem\_init}\OperatorTok{(\&}\NormalTok{empty}\OperatorTok{,} \DecValTok{0} \OperatorTok{,}\NormalTok{ N}\OperatorTok{);} \CommentTok{// buffer vide}
\NormalTok{sem\_init}\OperatorTok{(\&}\NormalTok{full}\OperatorTok{,} \DecValTok{0} \OperatorTok{,} \DecValTok{0}\OperatorTok{);} \CommentTok{// buffer vide}

\CommentTok{// Producteur}
\DataTypeTok{void}\NormalTok{ producer}\OperatorTok{(}\DataTypeTok{void}\OperatorTok{)} \OperatorTok{\{}
    \DataTypeTok{int}\NormalTok{ item}\OperatorTok{;} 
    \ControlFlowTok{while}\OperatorTok{(}\KeywordTok{true}\OperatorTok{)} \OperatorTok{\{}
\NormalTok{        item}\OperatorTok{=}\NormalTok{produce}\OperatorTok{(}\NormalTok{item}\OperatorTok{);} 
\NormalTok{        sem\_wait}\OperatorTok{(\&}\NormalTok{empty}\OperatorTok{);} \CommentTok{// attente d\textquotesingle{}une place libre }
\NormalTok{        pthread\_mutex\_lock}\OperatorTok{(\&}\NormalTok{mutex}\OperatorTok{);} 
        \CommentTok{// section critique }
\NormalTok{        insert\_item}\OperatorTok{();}
\NormalTok{        pthread\_mutex\_unlock}\OperatorTok{(\&}\NormalTok{mutex}\OperatorTok{);}
\NormalTok{        sem\_post}\OperatorTok{(\&}\NormalTok{full}\OperatorTok{);} \CommentTok{// une place remplie}
    \OperatorTok{\}}
\OperatorTok{\}}

\CommentTok{// Consommateur}
\DataTypeTok{void}\NormalTok{ consumer}\OperatorTok{(}\DataTypeTok{void}\OperatorTok{)} \OperatorTok{\{}
    \DataTypeTok{int}\NormalTok{ item}\OperatorTok{;} 
    \ControlFlowTok{while}\OperatorTok{(}\KeywordTok{true}\OperatorTok{)} \OperatorTok{\{}
\NormalTok{        sem\_wait}\OperatorTok{(\&}\NormalTok{full}\OperatorTok{);} \CommentTok{// attente d\textquotesingle{}une place remplie }
\NormalTok{        pthread\_mutex\_lock}\OperatorTok{(\&}\NormalTok{mutex}\OperatorTok{);} 
        \CommentTok{// section critique }
\NormalTok{        item }\OperatorTok{=}\NormalTok{ remove}\OperatorTok{(}\NormalTok{item}\OperatorTok{);}
\NormalTok{        pthread\_mutex\_unlock}\OperatorTok{(\&}\NormalTok{mutex}\OperatorTok{);}
\NormalTok{        sem\_post}\OperatorTok{(\&}\NormalTok{empty}\OperatorTok{);} \CommentTok{// une place libre}
    \OperatorTok{\}}
\OperatorTok{\}}
\end{Highlighting}
\end{Shaded}

\paragraph{Reader-writers}\label{reader-writers}

\begin{Shaded}
\begin{Highlighting}[]
\NormalTok{pthread\_mutex\_t mutex}\OperatorTok{;} 
\NormalTok{sem\_t db}\OperatorTok{;} \CommentTok{// accès à la db }
\DataTypeTok{int}\NormalTok{ readcount}\OperatorTok{=}\DecValTok{0}\OperatorTok{;} \CommentTok{// nombre de readers}

\NormalTok{sem\_init}\OperatorTok{(\&}\NormalTok{db}\OperatorTok{,}\NormalTok{ NULL}\OperatorTok{,} \DecValTok{1}\OperatorTok{)}

\DataTypeTok{void}\NormalTok{ writer}\OperatorTok{(}\DataTypeTok{void}\OperatorTok{)} \OperatorTok{\{}
    \ControlFlowTok{while}\OperatorTok{(}\KeywordTok{true}\OperatorTok{)} \OperatorTok{\{}
\NormalTok{        prepare\_data}\OperatorTok{();}
\NormalTok{        sem\_wait}\OperatorTok{(\&}\NormalTok{db}\OperatorTok{);}
        \CommentTok{// section critique, un seul writer à la fois }
\NormalTok{        write\_database}\OperatorTok{();} 
\NormalTok{        sem\_post}\OperatorTok{(\&}\NormalTok{db}\OperatorTok{);} 
    \OperatorTok{\}}
\OperatorTok{\}}

\DataTypeTok{void}\NormalTok{ reader}\OperatorTok{(}\DataTypeTok{void}\OperatorTok{)} \OperatorTok{\{}
    \ControlFlowTok{while}\OperatorTok{(}\KeywordTok{true}\OperatorTok{)} \OperatorTok{\{}
\NormalTok{        pthread\_mutex\_lock}\OperatorTok{(\&}\NormalTok{mutex}\OperatorTok{);} 
        \CommentTok{// section critique }
\NormalTok{        readcount}\OperatorTok{++;} 
        \ControlFlowTok{if} \OperatorTok{(}\NormalTok{readcount}\OperatorTok{==}\DecValTok{1}\OperatorTok{)} \OperatorTok{\{} 
            \CommentTok{// arrivée du premier reader }
\NormalTok{            sem\_wait}\OperatorTok{(\&}\NormalTok{db}\OperatorTok{);}
        \OperatorTok{\}}
\NormalTok{        pthread\_mutex\_unlock}\OperatorTok{(\&}\NormalTok{mutex}\OperatorTok{);} 
\NormalTok{        read\_database}\OperatorTok{();}
\NormalTok{        pthread\_mutex\_lock}\OperatorTok{(\&}\NormalTok{mutex}\OperatorTok{);} 
        \CommentTok{// section critique }
\NormalTok{        readcount}\OperatorTok{{-}{-};} 
        \ControlFlowTok{if}\OperatorTok{(}\NormalTok{readcount}\OperatorTok{==}\DecValTok{0}\OperatorTok{)} \OperatorTok{\{} 
            \CommentTok{// départ du dernier reader }
\NormalTok{            sem\_post}\OperatorTok{(\&}\NormalTok{db}\OperatorTok{);}
        \OperatorTok{\}}
\NormalTok{        pthread\_mutex\_unlock}\OperatorTok{(\&}\NormalTok{mutex}\OperatorTok{);} 
\NormalTok{        process\_data}\OperatorTok{();}
    \OperatorTok{\}}
\OperatorTok{\}}
\end{Highlighting}
\end{Shaded}

On assure bien que l'écrivain ne peut être en section critique en même
temps qu'un lecteur. On a un accès parallèle pour les écrivains. On
risque juste d'avoir un writer qui attend indéfiniment que tous les
écrivains se libèrent, problème de \textbf{famine}.

\subsubsection{\texorpdfstring{Variable
\texttt{volatile}}{Variable volatile}}\label{variable-volatile}

Quand on accède à une variable en C, C va mémoriser la valeur. On peut
s'assurer qu'il relise la valeur à chaque fois en ajoutant le mot-clé
\texttt{volatile}. \textbf{Ce n'est pas une solution pour se protéger
des écritures et accès concurrent !}

\subsubsection{Threads-safe}\label{threads-safe}

Une fonction est dite thread-safe si des threads en concurrences peuvent
utiliser correctement une même fonction. Si elle utilise des variables
globales ou statiques ❌.

\subsection{Les processus}\label{les-processus}

\emph{Définition}: Instructions à exécuter par les processeurs. Cela
peut avoir plusieurs contextes d'exécutions. Chaque processus à son
\textbf{espace mémoire spécifique}. Un segment \emph{stack} par contexte
d'exécution. Il a également des méta-données permettant au SE de le
contrôler.

\subsubsection{Librairies}\label{librairies}

On utilise souvent des librairies en C sous la forme de header
\texttt{math.h}. Un processus ne contient généralement pas le code des
librairies. On utilise donc les librairies partagées du SE (un peu comme
les fichiers midi).

\subsubsection{Création d'un processus}\label{cruxe9ation-dun-processus}

Les étapes:

\begin{enumerate}
\def\labelenumi{\arabic{enumi}.}
\tightlist
\item
  Le shell doit localiser le fichier exécutable dans le \texttt{PATH}
\item
  Demande au kernel de créer un nouveau processus
\item
  Le processus demande au kernel le chargement et l'exécution du code du
  fichier programme
\item
  On peut bloqué le shell et attendre de récupérer la valeur de retour
\end{enumerate}

On peut créer de nouveau processus via un \texttt{fork}. Cela fait des
\textbf{copies} distinctes qui ne partagent rien entre elles.
L'identifiant du processus va être différent. Le retour du père est 0 et
des fils \textgreater{} 0.

\begin{Shaded}
\begin{Highlighting}[]
\PreprocessorTok{\#include }\ImportTok{\textless{}stdio.h\textgreater{}}\PreprocessorTok{ }
\PreprocessorTok{\#include }\ImportTok{\textless{}stdlib.h\textgreater{}}\PreprocessorTok{ }
\PreprocessorTok{\#include }\ImportTok{\textless{}unistd.h\textgreater{}}

\DataTypeTok{int}\NormalTok{ g}\OperatorTok{=}\DecValTok{0}\OperatorTok{;} \CommentTok{// segment données}

\DataTypeTok{int}\NormalTok{ main }\OperatorTok{(}\DataTypeTok{int}\NormalTok{ argc}\OperatorTok{,} \DataTypeTok{char} \OperatorTok{*}\NormalTok{argv}\OperatorTok{[])} \OperatorTok{\{} 
    \DataTypeTok{int}\NormalTok{ l}\OperatorTok{=}\DecValTok{1252}\OperatorTok{;} \CommentTok{// sur la pile }
    \DataTypeTok{int} \OperatorTok{*}\NormalTok{m}\OperatorTok{;}     \CommentTok{// sur le heap}
\NormalTok{    m}\OperatorTok{=(}\DataTypeTok{int} \OperatorTok{*)}\NormalTok{ malloc}\OperatorTok{(}\KeywordTok{sizeof}\OperatorTok{(}\DataTypeTok{int}\OperatorTok{));} 
    \OperatorTok{*}\NormalTok{m}\OperatorTok{={-}}\DecValTok{1}\OperatorTok{;}

\NormalTok{    pid\_t pid}\OperatorTok{;} 
\NormalTok{    pid}\OperatorTok{=}\NormalTok{fork}\OperatorTok{();}
    
    \ControlFlowTok{if} \OperatorTok{(}\NormalTok{pid}\OperatorTok{=={-}}\DecValTok{1}\OperatorTok{)} \OperatorTok{\{}
    \CommentTok{// erreur à l\textquotesingle{}exécution de fork }
\NormalTok{    perror}\OperatorTok{(}\StringTok{"fork"}\OperatorTok{);} 
\NormalTok{    exit}\OperatorTok{(}\NormalTok{EXIT\_FAILURE}\OperatorTok{);}
    \OperatorTok{\}}
    \CommentTok{// pas d\textquotesingle{}erreur }
    \ControlFlowTok{if} \OperatorTok{(}\NormalTok{pid}\OperatorTok{==}\DecValTok{0}\OperatorTok{)} \OperatorTok{\{}
        \CommentTok{// processus fils }
\NormalTok{        l}\OperatorTok{++;}\NormalTok{ g}\OperatorTok{++;} \OperatorTok{*}\NormalTok{m}\OperatorTok{=}\DecValTok{17}\OperatorTok{;} 
\NormalTok{        printf}\OperatorTok{(}\StringTok{"Dans le processus fils g=}\SpecialCharTok{\%d}\StringTok{, l=}\SpecialCharTok{\%d}\StringTok{ et *m=}\SpecialCharTok{\%d\textbackslash{}n}\StringTok{"}\OperatorTok{,}\NormalTok{ g}\OperatorTok{,}\NormalTok{l}\OperatorTok{,*}\NormalTok{m}\OperatorTok{);} 
\NormalTok{        free}\OperatorTok{(}\NormalTok{m}\OperatorTok{);} 
        \ControlFlowTok{return}\OperatorTok{(}\NormalTok{EXIT\_SUCCESS}\OperatorTok{);} 
    \OperatorTok{\}} \ControlFlowTok{else} \OperatorTok{\{}
        \CommentTok{// processus père }
\NormalTok{        sleep}\OperatorTok{(}\DecValTok{2}\OperatorTok{);} 
\NormalTok{        printf}\OperatorTok{(}\StringTok{"Dans le processus père g=}\SpecialCharTok{\%d}\StringTok{, l=}\SpecialCharTok{\%d}\StringTok{ et *m=}\SpecialCharTok{\%d\textbackslash{}n}\StringTok{"}\OperatorTok{,}\NormalTok{g}\OperatorTok{,}\NormalTok{l}\OperatorTok{,*}\NormalTok{m}\OperatorTok{);}
\NormalTok{        free}\OperatorTok{(}\NormalTok{m}\OperatorTok{);} 
        \CommentTok{// ... }
        \ControlFlowTok{return}\OperatorTok{(}\NormalTok{EXIT\_SUCCESS}\OperatorTok{);}
    \OperatorTok{\}}
\OperatorTok{\}}
\end{Highlighting}
\end{Shaded}

Output:

\begin{verbatim}
Dans le processus fils g=1, l=1253 et *m=17 
Dans le processus père g=0, l=1252 et *m=-1
\end{verbatim}

On a aussi les \emph{PID} qui sont les identificateurs des processus. Le
PID 1 est le processus \emph{init} donc au démarrage du système.

\subsubsection{Flux Standards et de
Partage}\label{flux-standards-et-de-partage}

On a un partage entre processus père et fils de \emph{STDOUT} et
\emph{STDERR}. \emph{STDIN} est soit conservé ou délégué par le
processus père. Ils écrivent donc dans la même sortie.

\texttt{printf} ne fait pas d'appel bloquant mais il fait du
\textbf{buffering} donc n'imprime pas tout de suite. Ainsi il ne fait
qu'un seul appel à \texttt{write} par ligne !

\subsubsection{Fin d'un Processus}\label{fin-dun-processus}

Pour mettre fin à un processus, on utilise \texttt{exit()} ou un
\texttt{retur(0)} (implicite par le compilateur). On a toujours la
valeur du processus père et on peut spécifier des actions à faire à la
terminaison \texttt{atexit()} (on spécifie une fonction qu'on veut
réaliser). \texttt{exit()} fait un appel système \texttt{exit}.

Pour récupérer une valeur de retour, on utilise:

\begin{Shaded}
\begin{Highlighting}[]
\PreprocessorTok{\#include }\ImportTok{\textless{}sys/types.h\textgreater{}}\PreprocessorTok{ }
\PreprocessorTok{\#include }\ImportTok{\textless{}sys/wait.h\textgreater{}}

\NormalTok{pid\_t waitpid}\OperatorTok{(}\NormalTok{pid\_t pid}\OperatorTok{,} \DataTypeTok{int} \OperatorTok{*}\NormalTok{status}\OperatorTok{,} \DataTypeTok{int}\NormalTok{ options}\OperatorTok{);}
\CommentTok{// pid:     identifiant fils}
\CommentTok{// status:  entier qui contiendra la valeur}
\CommentTok{// options: des options}
\end{Highlighting}
\end{Shaded}

\subsubsection{Processus Orphelin et
Zombie}\label{processus-orphelin-et-zombie}

\begin{itemize}
\tightlist
\item
  \emph{Orphelin}: Si un processus père n'attend pas un processus fils,
  il sera rattaché au PID 1.
\item
  \emph{Zombie} : Si un processus à terminer mais que sa valeur de
  retour n'a jamais été récupérée, le kernel doit la stocker.
\end{itemize}

\subsubsection{Exécution d'un
programme}\label{exuxe9cution-dun-programme}

Pour éviter d'être limité à des exécutions Père-Fils où le fils à le
même code que celui du père et une copie de ses données, on peut
remplacer la mémoire de ce processus via celle dans un fichier
exécutable comme ceci:

\begin{Shaded}
\begin{Highlighting}[]
\PreprocessorTok{\#include }\ImportTok{\textless{}unistd.h\textgreater{}}\PreprocessorTok{ }

\DataTypeTok{int}\NormalTok{ execve}\OperatorTok{(}\DataTypeTok{const} \DataTypeTok{char} \OperatorTok{*}\NormalTok{path}\OperatorTok{,} \DataTypeTok{char} \OperatorTok{*}\DataTypeTok{const}\NormalTok{ argv}\OperatorTok{[],} \DataTypeTok{char} \OperatorTok{*}\DataTypeTok{const}\NormalTok{ envp}\OperatorTok{[]);}
\CommentTok{// path: l\textquotesingle{}espace mémoire à utiliser}
\CommentTok{// argv: les arguments à passer}
\CommentTok{// envp: les informations vis{-}à{-}vis de l\textquotesingle{}environnement}
\end{Highlighting}
\end{Shaded}

Pour consulter tous les processus en cours, on utilise \texttt{ps},
\texttt{top} (gestionnaire de tâche), \texttt{pstree} (voir les
relations père-fils). Tout cela va provenir de \texttt{/proc} (pseudo
système de fichier).
