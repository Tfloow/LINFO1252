\href{../README.md}{\textless--}

\begin{center}\rule{0.5\linewidth}{0.5pt}\end{center}

\section{Cours 5}\label{cours-5}

\begin{itemize}
\tightlist
\item
  \hyperref[cours-5]{Cours 5}

  \begin{itemize}
  \tightlist
  \item
    \hyperref[service]{Service}

    \begin{itemize}
    \tightlist
    \item
      \hyperref[interface]{Interface}
    \item
      \hyperref[services-aux-concepteurs-dapplications]{Services aux
      concepteurs d'applications}
    \item
      \hyperref[services-aux-applications]{Services aux applications}
    \item
      \hyperref[accuxe8s-aux-services-du-se]{Accès aux services du SE}
    \end{itemize}
  \item
    \hyperref[architecture-des-systuxe8mes-dexploitation]{Architecture
    des systèmes d'exploitation}

    \begin{itemize}
    \tightlist
    \item
      \hyperref[objectifs-et-contraintes-de-mise-en-oeuvre-dun-se]{Objectifs
      et contraintes de mise en oeuvre d'un SE}
    \item
      \hyperref[ms-dos]{MS-DOS}
    \item
      \hyperref[unix]{UNIX}
    \item
      \hyperref[macro--et-micro-noyaux]{Macro- et Micro-noyaux}
    \item
      \hyperref[fiabilituxe9-des-se]{Fiabilité des SE}
    \item
      \hyperref[duxe9marrage-du-se]{Démarrage du SE}
    \end{itemize}
  \end{itemize}
\end{itemize}

\subsection{Service}\label{service}

Un système d'exploitation fournit des services aux utilisateurs (aussi
distant), aux développeurs et programmes qui s'exécutent.

\subsubsection{Interface}\label{interface}

\begin{itemize}
\tightlist
\item
  Interface utilisateur: GUI ou terminal
\item
  Interface de traitement par lot: comme Inginious, on a un ensemble de
  tâches qui s'exécutent groupe par groupe. (punch card)
\end{itemize}

On retrouve la dernière interface sur les HPC et autres types de super
computer.

\paragraph{Sous linux}\label{sous-linux}

On a un système pour le batch (traitement par lot).

\begin{itemize}
\tightlist
\item
  \texttt{at}: lance une commande à un moment spécifique.
\item
  \texttt{crontab}: gérer des tâches récurrentes.
\end{itemize}

\subsubsection{Services aux concepteurs
d'applications}\label{services-aux-concepteurs-dapplications}

On facilite le déploiement d'applications sur d'autres systèmes que
celui de la machine du dev via: - \textbf{Linker}: Assemble différents
fichiers \emph{objets} en un \emph{exécutable unique} - \texttt{ld} sur
Linux - \texttt{-static} avec \texttt{gcc} pour avoir l'ensemble des
libraires - \textbf{Loader}: au démarrage d'un programme. -
\texttt{ld.so} libraire et support du kernel Linux 1. Initialise
l'espace mémoire 2. Pré-charge les segments text, data, environment 3.
Localiser et charger les libraires dynamiques

Il peut aussi gérer les erreurs. ex: \emph{memory dump} pour avoir
l'état de la mémoire avant le crash. Ce \emph{memory dump} peut être
déclenché si on fait une division par 0 ou accès non autorisé à certains
segments de la mémoire.

On peut aussi facilement débugger via \texttt{gdb} et ses breakpoints.
Les breakpoints sont en réalité des \emph{trap} qui fait une
interruption logicielle !

\subsubsection{Services aux
applications}\label{services-aux-applications}

Le SE gère les entrées sorties mais il y a quelques règles: - Une app
\textbf{ne peut pas} envoyer d'instructions aux gestionnaires de
périphériques - \textbf{Ne peut pas} traiter les interruptions -
Hétérogénéité des périphériques eux-mêmes - SE fournit une abstraction
unique pour une \emph{classe} de périphérique

Un des buts du SE est de faire \textbf{correspondre} des abstractions
\emph{haut niveau} et des opérations \emph{bas niveau}. On a des drivers
qui permettent de contrôler le matériel.

Un SE va partager les ressources. Il va faire: - Maximisation de
l'utilisation - Équité d'accès - Isolation

Il fournit aussi d'autres services spécifiques: - Allocations des
ressources: - Exclusive ou non (typique des ports) - Contrôle générique
ou spécifique - Protection et contrôle d'accès

\subsubsection{Accès aux services du
SE}\label{accuxe8s-aux-services-du-se}

\begin{itemize}
\tightlist
\item
  Utilisation d'utilitaire systèmes
\item
  Appel direct des fonctions du noyau
\item
  Librairie Standard
\end{itemize}

On a une API par le noyau pour les \emph{appels système}: - A un numéro
précis - Point d'entrée unique pour accéder aux fonctions du noyau

\paragraph{Fonctionnement}\label{fonctionnement}

\begin{enumerate}
\def\labelenumi{\arabic{enumi}.}
\tightlist
\item
  Placer les arguments dans la pile
\item
  Sauvegarde l'adresse de retour sur la pile
\item
  Modifier \texttt{\%eip} pour que la prochaine instruction à exécuter
  soit notre fonction
\item
  Fonction prend ses arguments
\item
  Sauvegarde son résultat à un endroit (\texttt{\%eax} par convention en
  IA32)
\item
  Fonction récupère l'adresse de retour sur la pile et modifie
  \texttt{\%eip} pour retourner à la fonction appelante.
\end{enumerate}

\begin{figure}
\centering
\includegraphics{image-12.png}
\caption{Alt text}
\end{figure}

On change de mode d'exécution. On change d'espace mémoire !

\paragraph{Réalisation concrète}\label{ruxe9alisation-concruxe8te}

Problème 1: placer des arguments qui seront accessibles par le noyau. On
va le mettre soit dans: - Un espace mémoire fixe dédié (data) - Sur la
pile - Registres (\texttt{\%ebx} pour premier argument puis
\texttt{\%ecx})

Problème 2: adresse de retour - Pointeur de programme sauvegardé sur la
pile. - Restauré au retour

Problème 3: appel effectif - Passer en mode protégé - Instructions
spéciales (IA32 \texttt{int\ 0x80} crée une interruption) ou simplement
\texttt{syscall}. - Point d'entrée du kernel est connu par le
processeur.

\begin{figure}
\centering
\includegraphics{image-13.png}
\caption{Alt text}
\end{figure}

Problème 4: récupération du code de retour 1. Opération autorisée et
paramètre corrects: - Résultat mis dans le registre \texttt{\%eax} -
Instruction de retour au mode utilisateur, dépile l'adresse de retour et
positionne le compteur de programme 2. Erreur ou opération non autorisée
- Positionne la variable d'environnement \texttt{errno} - Retourne aux
processus parent (ex: shell)

\paragraph{Appels systèmes}\label{appels-systuxe8mes}

\begin{itemize}
\tightlist
\item
  Librairie standard: s'exécute dans le même espace mémoire que le
  programme utilisateur. \texttt{printf(3)} va forcément utiliser
  \texttt{write(2)} ou autres opérations bas niveau.
\item
  \texttt{strace(1)} permet de tracer les appels systèmes utilisés par
  un processus
\end{itemize}

.

\begin{figure}
\centering
\includegraphics{image-14.png}
\caption{Alt text}
\end{figure}

\paragraph{Hello world}\label{hello-world}

\begin{Shaded}
\begin{Highlighting}[]
\PreprocessorTok{\#include }\ImportTok{\textless{}stdio.h\textgreater{}}
\PreprocessorTok{\#include }\ImportTok{\textless{}stdlib.h\textgreater{}}

\DataTypeTok{int}\NormalTok{ main}\OperatorTok{(}\DataTypeTok{int}\NormalTok{ argc}\OperatorTok{,} \DataTypeTok{char} \OperatorTok{*}\NormalTok{argv}\OperatorTok{[])\{}
\NormalTok{   printf}\OperatorTok{(}\StringTok{"Hello, world! }\SpecialCharTok{\%d\textbackslash{}n}\StringTok{"}\OperatorTok{,}\KeywordTok{sizeof}\OperatorTok{(}\DataTypeTok{int}\OperatorTok{));}
   \ControlFlowTok{return}\NormalTok{ EXIT\_SUCCESS}\OperatorTok{;}
\OperatorTok{\}}
\end{Highlighting}
\end{Shaded}

Traduction:

\begin{verbatim}
$ strace ./helloworld_s
execve("./helloworld_s", ["./helloworld_s"], [/* 21 vars */]) = 0
uname({sys="Linux", node="precise32", ...}) = 0
brk(0)                                  = 0x9e8b000
brk(0x9e8bd40)                          = 0x9e8bd40
set_thread_area({entry_number:-1 -> 6, base_addr:0x9e8b840, limit:1048575, seg_32bit:1, 
contents:0, read_exec_only:0, limit_in_pages:1, seg_not_present:0, useable:1}) = 0
brk(0x9eacd40)                          = 0x9eacd40
brk(0x9ead000)                          = 0x9ead000
fstat64(1, {st_mode=S_IFCHR|0620, st_rdev=makedev(136, 0), ...}) = 0
mmap2(NULL, 4096, PROT_READ|PROT_WRITE, MAP_PRIVATE|MAP_ANONYMOUS, -1, 0) = 0xb778a000
write(1, "Hello, world! 4\n", 16Hello, world! 4
)       = 16
exit_group(0)                           = ?
\end{verbatim}

\subsection{Architecture des systèmes
d'exploitation}\label{architecture-des-systuxe8mes-dexploitation}

\begin{figure}
\centering
\includegraphics{image-15.png}
\caption{Alt text}
\end{figure}

\subsubsection{Objectifs et contraintes de mise en oeuvre d'un
SE}\label{objectifs-et-contraintes-de-mise-en-oeuvre-dun-se}

\begin{itemize}
\tightlist
\item
  Pas de modèle unique et parfait
\item
  Contraintes et objectifs:

  \begin{itemize}
  \tightlist
  \item
    Contraintes du matériel
  \item
    Performance et coût d'implémentation
  \item
    Consommation de ressources
  \item
    Facilité d'évolution et d'adaptation
  \item
    Facilité de maintenance et d'atteinte de fiabilité
  \end{itemize}
\end{itemize}

\subsubsection{MS-DOS}\label{ms-dos}

SE des systèmes \emph{IBM-PC} pas basé sur UNIX, c'est Microsoft.

Objectifs: - Mono-utilisateur et mono-application (Pas de temps partagé)
- Visant des processeurs \textbf{ne supportant pas} les modes
utilisateur/protégé (Pas d'isolation) - Contrainte forte sur
l'utilisation de la mémoire (coutait extrêmement cher à l'époque)

Système avec une vision \textbf{monolithique} pour éviter de consommer
trop de mémoire.

\begin{figure}
\centering
\includegraphics{image-16.png}
\caption{Alt text}
\end{figure}

\begin{figure}
\centering
\includegraphics{image-17.png}
\caption{Alt text}
\end{figure}

\subsubsection{UNIX}\label{unix}

\paragraph{Monolithe}\label{monolithe}

Plus vieux que \hyperref[ms-dos]{MS-DOS} mais pensé pour des ordinateurs
à plus grande capacité. - Multi utilisateur et temps partagé -
Processeur supportant les deux modes - Séparation claire entre
\emph{noyau} et \emph{programmes utilisateurs}

Organisation originelle: \emph{monolithe}: - 1 seul programme sur 1
seule couche, met en oeuvre tous les appels systèmes - Difficile à
étendre, adapter, débugger

\begin{figure}
\centering
\includegraphics{image-18.png}
\caption{Alt text}
\end{figure}

\paragraph{Couches}\label{couches}

On va rajouter de la structure pour pallier aux soucis lié à
\hyperref[monolithe]{unix monolithique}

Les couches: 1. Matériel 2. Drivers de périphériques 3. Abstractions des
plus en plus haut niveau, jusqu'aux appels systèmes 4. Interface
utilisateur

\begin{figure}
\centering
\includegraphics{image-19.png}
\caption{Alt text}
\end{figure}

Il y a des avantages et inconvénients !

Avantages: - Isolations des fonctionnalités - Facilité de portage

Inconvénients: - Surcoût à l'exécution des appels système - Difficile de
décider d'une structure purement hiérarchique. - Il y a une
interdépendance entre les fonctions du SE

\paragraph{Structure en Modules}\label{structure-en-modules}

\begin{figure}
\centering
\includegraphics{image-20.png}
\caption{Alt text}
\end{figure}

Utilisé le plus souvent: Linux, Solaris, Windows

On a un coeur qui est un monolithe ou qui a peu de couche. - Gestion bas
niveau de la mémoire - Gestion des processus

Le reste est sous forme de modules: - Chargés et décharges de l'espace
mémoire du noyau \emph{dynamiquement} - Seulement lorsque nécessaire -
Ex: quand on met une clé usb. Va charger \texttt{exFAT} si clé venant de
Windows

Avantages: - Spécialisation d'une SE pour un environnement donné: -
Versatilité de Linux - Interactions possibles entre les modules en
conservant la séparation de code et de mémoire. - Analogie possible avec
programmation orientée objet

Inconvénients: - Surcoût (négligeable sur les systèmes modernes) -
Intégration de modules écrits par des tiers dans l'espace mémoire du
noyau - potentiels bugs et fautes

Gestions des modules sous Linux (seulement si \texttt{root}): -
\texttt{lsmod}: liste les modules présents - \texttt{modprobe}:
ajoute/supprime un module - \texttt{modinfo}: information sur un module

\subsubsection{Macro- et Micro-noyaux}\label{macro--et-micro-noyaux}

\paragraph{Macro-noyaux}\label{macro-noyaux}

Tous les modes précédents placent l'intégralité des fonctions du SE dans
le noyau. - Toutes s'exécutent en mode protégé - Accès complet à la
mémoire et aux instructions sensibles - Crash du module = \textbf{crash
du système}

Le kernel est en pérille, corruption des données, faille pour les
hackers.

\paragraph{Micro-noyaux}\label{micro-noyaux}

Séparation entre un noyau minimaliste - Gestion basique de la mémoire -
Gestion des processus légers (threads) - Gestion de la communication
entre processus (IPC)

Le reste est mis en oeuvre par des processus en mode utilisateur (même
les drivers)

Réduit les bugs de modules et évitent les crash complets tout en gardant
les propriétés d'isolation.

\begin{figure}
\centering
\includegraphics{image-21.png}
\caption{Alt text}
\end{figure}

Inconvénients: - En macro: appel de module dans un même espace mémoire
donc très rapide. En micro: il faut faire des appels systèmes - Copie de
l'argument de l'appelant vers le noyau - Copie dans l'espace mémoire de
l'appelé - Changement de contexte (sauvegarde du contexte et
restauration\ldots) - \ldots{} et rebelote dans l'autre sens pour la
valeur de retour

Donc on a longtemps abandonnés les micro-noyaux (abandon dans Windows
NT) MAIS: - Progrès sensibles dans la mise en oeuvre des appels systèmes
- Passage de message en mode zero-copy - Mac OS et iOS proche d'un
micro-noyau

\subsubsection{Fiabilité des SE}\label{fiabilituxe9-des-se}

Souvent beaucoup de temps nécessaires pour corriger des fautes.
Énormément de fautes dans les drivers et compliqués à débugger et
trouver parfois.

\paragraph{Certification formelle}\label{certification-formelle}

Pour les systèmes embarqués critiques on réalise des
\textbf{spécification formelle}. On modélise \emph{mathématiquement}
qu'un système est fiable. C'est le cas pour seL4. Mais seulement
possible si le code est petit.

\subsubsection{Démarrage du SE}\label{duxe9marrage-du-se}

Démarrage et reset de la machine: interruption de mise à zéro du
processeur. 2 phases:

\begin{enumerate}
\def\labelenumi{\arabic{enumi}.}
\tightlist
\item
  Programme de démarrage (bootstrap) stocké dans une ROM. Charge ainsi
  le bootloader
\item
  Bootloader lit les systèmes de fichiers pour charger l'image du noyau
  en mémoire (\texttt{/boot/vmlinuz-3.12.0-32-generic}) en gros le GRUB.
\end{enumerate}

\begin{center}\rule{0.5\linewidth}{0.5pt}\end{center}

\href{../README.md}{\textless--}
